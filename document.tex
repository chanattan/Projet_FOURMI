\documentclass[11pt]{article}
\usepackage{xcolor}
\usepackage[utf8]{inputenc}
\usepackage[french]{babel}
\usepackage[T1]{fontenc}
\usepackage{verbatim}

\usepackage[a4paper]{geometry}

\title{Programmation projet fourmi\\Vendredi avant 19h très important\\Tournoi après les vacances et des lots à gagner}
\author{L3 SIF\\\\Chanattan Sok\\chanattan.sok@ens-rennes.fr}
\begin{document}
\maketitle
\section{Objectifs}
\begin{itemize}
	\item Design le langage de programmation
	\item Ecrire le compilateur
	\item Créer une stratégie
\end{itemize}
\section{Compilateur}
Fichier source en entrée (ocaml) $\to$ Analyse lexicale (techniques langages formels pour découper en mots/tokens le code) $\to$ Analyse syntaxique (transformer en AsT les tokens) $\to$ Analayse sémantique (typage correct? sens?) $\to$ (Optimisation (code mort?) registres?) $\to$ Génération de code\\\\
La partie : Analyse lexicale, Analyse syntaxique est donnée.\\
Analyser lexicale : Lexer\\
Analyse syntaxique : Parser donne \textbf{langage} $\to$ lang.grammar\\
A faire génération de code (langage fourmi) et faire l'optimisation (donner priorité à une certaine fourmi??).\\
Génération code donne \textbf{antx.ml}.
\end{document}